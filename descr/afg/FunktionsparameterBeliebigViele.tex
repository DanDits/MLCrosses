Unter \href{https://docs.python.org/3/tutorial/controlflow.html\#defining-functions}{https://docs.python.org/3/tutorial/controlflow.html\#defining-functions} finden Sie in den Abschnitten 4.6 und 4.7 bis 4.7.4 weitere Informationen zu Funktionen und Parametern.

Nutzen Sie die dort enthaltenen Informationen zum Bearbeiten dieser Aufgabe.

\begin{enumerate}
    \item 
        Schreiben Sie eine Funktion \lstinline{addiere}, die beliebig viele Positionsparameter erwartet und die Summe aller übergebenen Argumente zurückgibt.
        
    \item
        Finden Sie einen Weg, der Funktion \lstinline{addiere} eine Liste von Zahlen zu übergeben, die der Benutzer über die Tastatur einliest. Ein Programmlauf soll folgendermaßen aussehen:
        \begin{lstlisting}
Geben Sie Zahlen ein (einfach Enter, um die Eingabe zu beenden)
Zahl: 7
Zahl: 2
Zahl: 1
Zahl: 8
Zahl:
Die Summe der eingegebenen Zahlen ist 18
        \end{lstlisting}

    \item
        Erstellen Sie eine Funktion \lstinline{my_print}, die dieselbe Schnittstelle wie die eingebaute Funktion \lstinline{print} besitzt und diese mit den übergebenen Argumenten aufruft. Außerdem soll sie vor der Ausgabe mit \lstinline{print} die Anzahl der übergebenen  Positionsargumente ausgeben.
        \begin{lstlisting}
>>> my_print("Ich", "rufe", "print", "mit", "sechs", "Argumenten")
Anzahl Positionsargumente: 6
Ich rufe print mit sechs Argumenten
        \end{lstlisting}
        \hint{Informieren Sie sich dazu über die genaue Schnittstelle von \lstinline{print}.}
        
    \item
        Ausgehend von einer Funktion \lstinline{f} betrachten Sie die folgenden Funktionsaufrufe.
        \begin{multicols}{2}
\begin{enumerate}[(A)]
    \item \lstinline{f(1)}
    \item \lstinline{f(1,2)}
    \item \lstinline{f(c=9)}
    \item \lstinline{f(1,2,3)}
    \item \lstinline{f(1,2,c=9)}
    \item \lstinline{f(1,2,3,4,5,6,7,8)}
\end{enumerate}
    \end{multicols}\vspace{-0.5em}
        Prüfen Sie, für welche der folgenden Funktionsdefinitionen welche der obigen Aufrufe funktionieren und interpretieren Sie die Ausgaben.
    \begin{multicols}{2}
\begin{enumerate}
    \item \lstinline{def f(a,b,c): print(a,b,c)}
    \item \lstinline{def f(a,b,c=5): print(a,b,c)}
    \item \lstinline{def f(a,*b,c=5): print(a,b,c)}
    \item \lstinline{def f(a,*b,c): print(a,b,c)}
    \item \lstinline{def f(a=6,*b,c): print(a,b,c)}
    \item \lstinline{def f(a,*,c): print(a,c)}
    \item \lstinline{def f(a=6,*,c): print(a,c)}
    \item \lstinline{def f(*,c): print(c)}
\end{enumerate}
    \end{multicols}        
\end{enumerate}

