Laden Sie sich die Datei \lstinline{matrix_vector_data.zip} aus dem ILIAS herunter.

Ziel der Aufgabe ist es, das Programm in \lstinline{matrix_vector.py} so zu ergänzen, dass es als Ausgabe den Inhalt von \lstinline{matrix_vector_ausgabe.txt} erzeugt. Bearbeiten Sie dazu die Aufgaben (a) und (b) in der Datei \lstinline{la.py}.
\vspace{-0.8em}
\begin{enumerate}
    \item
        Erstellen Sie eine Klasse \lstinline{Vector}, die einen Vektor im $\R^n$ modelliert. Jedes Objekt der Klasse \lstinline{Vector} soll ein Attribut \lstinline{_data} vom Datentyp \lstinline{list} besitzen, um die Einträge des Vektors zu speichern.
        
        Die Länge des Vektors soll über ein Argument festgelegt werden, das dem Konstruktor übergeben wird.
        
        Implementieren Sie die folgenden Methoden in der Klasse \lstinline{Vector}:
        \begin{multicols}{3}
            \begin{enumerate}
                \item \lstinline{size(self)}
                \item \lstinline{get_component(self, i)}
                \item \lstinline{as_string(self)}
                \item \lstinline{set_component(self, i, scalar)}
                \item \lstinline{normalize(self)}
            \end{enumerate}
        \end{multicols}\vspace{-0.5em}
        Dabei ist \lstinline{i} eine natürliche und \lstinline{scalar} eine reelle Zahl.
        
        \hint{Ein Vektor wird \emph{normalisiert}, indem er durch seine Länge geteilt wird.}

    \item
        Erzeugen Sie eine Klasse \lstinline{Matrix}, die eine Matrix in $\R^{m \times n}$ modelliert. Jede Instanz soll ein Attribut \lstinline{_rows} besitzen, das die Zeilen der Matrix als Instanzen der Klasse \lstinline{Vector} in einer Liste verwaltet.
        
        Die Dimensionen der Matrix sollen dem Konstruktor als Argumente übergeben werden.
        
	Implementieren Sie die folgenden Methoden in der Klasse \lstinline{Matrix}:
        \begin{multicols}{2}
            \begin{enumerate}
                \item \lstinline{size(self)}
                \item \lstinline{get_component(self, i, j)}
                \item \lstinline{get_row(self, i)}
                \item \lstinline{as_string(self)}
                \item \lstinline{set_component(self, i, j, scalar)}
                \item \lstinline{set_row(self, i, vector)}
            \end{enumerate}
        \end{multicols}


        
    \item
        Implementieren Sie mithilfe dieser beiden Klassen die Vektoriteration, die für eine Matrix $A \in \R^{n\times n}$ folgendermaßen funktioniert:
        \begin{enumerate}
            \item Wähle $x_0 \in \R^n$, $N \in \N$, $i=0$
            \item Solange $i < N\colon$ 
            
                \qquad $x_{i+1} := \frac{1}{|Ax_i|_2} Ax_i$, $i := i+1$
        \end{enumerate}
        Nutzen Sie die Vektoriteration, um eine eine Näherung des Eigenvektors zum größten Eigenwert von $A = \left(\begin{smallmatrix} 1 & 2 & 3 \\ 3 & 2 & 1 \\ 1 & 2 & 1 \end{smallmatrix}\right)$ zu ermitteln, wobei Sie $x_0 = (1,1,1)^\top$ als Startwert verwenden.
\end{enumerate}
