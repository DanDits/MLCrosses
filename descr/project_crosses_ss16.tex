\documentclass[10pt,a4paper,twocolumn,landscape,fleqn]{article}

\usepackage{ngerman,url}
\usepackage[utf8]{inputenc}
\usepackage{ueb}
\usepackage{amsmath,amssymb}
\usepackage{graphicx}
\usepackage{paralist}
\usepackage{multicol}
\usepackage{listings}
% \usepackage{todonotes}
\def\todo#1{{\color{red}TODO: #1}}

\RequirePackage{myDef}

\newcommand{\blattnr}{5}

\usepackage{mcode}
\usepackage[colorlinks=true,
            pdftitle={(UB \blattnr) Einf\"uhrung in Python},
            pdfauthor={Johannes Ernesti},
            pdfpagemode={UseOutlines},pdfstartview={Fit},
            linkcolor=blue,urlcolor=blue,
            pdfsubject={Einf\"uhrung in Python}]{hyperref}
\renewcommand\labelenumi{(\alph{enumi})}

\newcommand{\points}[1]{\text{\bf\emph{(#1 P.)}}}

\lstset{language=Python,
    aboveskip=0.5em,
    otherkeywords={nonlocal},
%     belowcaptionskip=0em,
    belowskip=0em,
}

\begin{document}
\kopf{Evaluationshelfer}{}
\newcounter{aufgabenr}
\setcounter{aufgabenr}{0}
\newcounter{paufgabenr}
\setcounter{paufgabenr}{1}

\inputauf{\stepcounter{aufgabenr}\theaufgabenr}{ProjektGrundriss}{Um was geht es}{}
\inputauf{\stepcounter{aufgabenr}\theaufgabenr}{ReferenzBogen}{Referenzbogen erstellen}{}
\inputauf{\stepcounter{aufgabenr}\theaufgabenr}{KaestchenExtrahieren}{K"astchen finden}{}
\inputauf{\stepcounter{aufgabenr}\theaufgabenr}{NeuronalesNetzwerk}{Neuronales Netzwerk}{}

\vspace{-1em}
\end{document}
 
